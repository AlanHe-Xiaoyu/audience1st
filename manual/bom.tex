
\section{Box Office Manager's Guide}
\label{chap:bom}

Whereas Box Office Agents can handle ticket sales and reservations for
performances, the Box Office Manager is the one who \emph{causes tickets
  and reservations to become available} for shows.  The general process
of entering a show into the system, associating performance dates with
it, and indicating what tickets are redeemable for each performance is
collectively called \emph{listing the show}.

It's important to realize that, in general, listing a show can involve
up to four steps:

\begin{enumerate}
\item[Add the show] Enter the overall information for the show---start
  and end dates, default seating capacity, etc.  This step can be
  skipped if just adding or removing performance dates from an existing
  show. 
\item[Add show dates] Associate performance dates with the show (or
  remove performance dates).  Information such as the deadline for
  advance sales cutoff and the specific house capacity can be overridden
  for each show date.
\item[Indicate valid vouchers] Associate particular ticket types
  (General Admission, Subscriber, etc.) with the performance.
  \emph{Without this step, the performances are listed but no vouchers
    can be redeemed for them.}
\end{enumerate}

In addition, the Box Office Manager can define new types of vouchers
(tickets) such as discount tickets, promotions, limited-capacity
tickets, etc.  In addition to \emph{regular} vouchers, the Box Office
Manager can set up \emph{bundle} vouchers; a bundle is a collection of
single tickets that is offered to the customer as a unit, for example, a
subscription or series package.

The next section covers how to set up both regular and bundle vouchers.
In general, you would set up most basic voucher types once only, and
only occasionally add new voucher types when you want to add a new
promotion or ticket type.

The subsequent three sections cover the three steps in listing a show:
entering show information (section~\ref{sec:addingshows}), adding
performance dates for a show (section~\ref{sec:addingshowdates}), and
indicating voucher validity for each performance date
(section~\ref{sec:validvouchers}).  

\subsection{Vouchers and Bundles}
\label{sec:vouchertypes-details}

There are two types of vouchers.  A \emph{regular} voucher is a single
seat or ticket to a single performance.  A \emph{bundle} is a package
that includes several regular vouchers.  For example, we may
offer a special 2-for-1 promo or kid-free-with-adult promo that is
limited to specific performances; to implement this you would create a
new voucher type representing this promo and then mark the voucher type
as valid only for certain performances.

\subsubsection{Creating a New Regular Voucher}
\label{sec:addingregularvouchertypes}


When a new voucher type is set up, whether or not it is a bundle,
additional properties can be specified for it:

\begin{itemize}
\item[Expiration] A date after which the voucher expires and can no
  longer be used.  Useful for time-limited promotions.
\item[Is subscriber] Does the purchase of this voucher qualify the
  purchaser as a ``subscriber''?  Useful for distinguishing bundles
  corresponding to subscriptions from bundles corresponding to promos;
  however, a non-bundle voucher can also be designated as
  subscriber-qualifying. 
\item[Changeable] Specifies whether the voucher is tied to a particular
  show date or can be used to reserve for a variety of dates. This
  attribute is not directly settable by the BOM; it's set by the system
  when the voucher is created.  Specifically, vouchers that are
  pre-purchased as part of subscriptions or promos are generally
  changeable; vouchers corresponding to tickets purchased for a
  particular performance are generally not.
\item
\end{itemize}


\subsubsection{Adding a New Bundle Voucher Type}
\label{sec:addingbundlevouchertypes}


A bundle is itself a voucher that cannot be ``redeemed'' for anything
but serves as a placeholder.  For example, if a patron buys a
subscription containing 3 vouchers good for any performance, they would
actually see 4 vouchers in their account.  The first is a nonredeemable
``voucher'' that records the type of bundle purchase.  The remaining 3
are the actual vouchers that can be redeemed for seats.  This makes it
easy to figure out how many people took advantage of a particular
bundle promotion even if the promotion includes ``regular'' tickets.

In general, to define a bundle voucher, you \emph{first} have to define the
regular vouchers that the bundle is going to include.

(To be written)

\subsection{Shows and Show Dates}

A \emph{show} is a series of 1 or more performances of the same piece.
In general, to list a show you must first enter the general information
about the production (opening and closing dates, house capacity, etc.),
then enter specific information about each show date.

\subsubsection{Creating Shows}
\label{sec:addingshows}


\subsubsection{Changing Show Details}
\label{sec:changingshowdetails}

\subsubsection{Adding Show Dates}
\label{sec:addingshowdates}

\subsection{Managing Voucher Validity For Show Dates}

To allow a particular type of voucher (e.g. ``General Admission'') to be
redeemed for a particular performance, you must indicate that it is a
\emph{valid voucher} for that performance.

\textbf{Important.} When a voucher type is first created, it is
\emph{not} automatically redeemable for any particular show.  Its
validity for specific performances must be explicitly established as
described in this section.

\subsection{Setting What Tickets Can Be Redeemed for a Performance}
\label{sec:validvouchers}

Note that a bundle voucher cannot be made redeemable, but the regular
vouchers included in the bundle can be.  For example, if a ``Family
Promo'' bundle voucher includes 2 ``Adult'' and 2 ``Child'' tickets, the
ticket types ``Adult'' and ``Child'' can be made redeemable, but the
bundle ``Family Promo'' cannot be.

\subsubsection{Adding Valid Vouchers to a Performance}
\label{sec:addingvalidvouchers}

Adding valid voucher types for a performance makes those voucher types
available for purchase or reservation for that performance.

To make a particular regular (nonbundle) voucher type valid for
redemption to a given performance, first be sure that the voucher type
exists (see section~\ref{sec:addingregularvouchertypes}) and the show
date for which you want to allow redemption exists (see
section~\ref{sec:addingshowdates}).  Navigate to the list of show dates
for the show in question (section~\ref{sec:addingshowdates}.

Next to each show date is the list of voucher types currently redeemable
for that performance.  (See figure XXX.)  If the voucher type you want
to make valid for the performance isn't listed there, click \emph{Add
  New}.  (If you \emph{do} see the voucher type listed, but you want to
change any validity properties such as redemption limits or sales
cutoff, click on the name of the voucher type and see
section~\ref{sec:changingvalidvoucher}.) 


\subsubsection{Changing Existing Valid Vouchers}
\label{sec:changingvalidvoucher}

To be written
