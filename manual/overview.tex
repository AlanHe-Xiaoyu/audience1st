\section{System Overview}

Throughout this document, wherever VENUE appears in all uppercase, you
should substitute for it the name of the venue as encoded by Audience1st
for your particular installation.

For example, if your venue is The Little Theater and the
Audience1st-assigned name for the venue is \verb+littletheater+, then a
URL that appears in this manual as
\url{http://www.audience1st.com/VENUE/store} should be translated to
\url{http://www.audience1st.com/littletheater/store} for use with your
venue.  You'll most often see this when the manual refers to URL's (Web
addresses) or email addresses.

\subsection{Patrons, Vouchers,  Reservations, Bundles, and Tickets}
\label{sec:overview-vouchers}

The \emph{patron} is the unit to which most operations are indexed.
Ticket vouchers (see below) are linked to
patrons; donations are linked to patrons.

\af operates on the basis of ticket \T{vouchers}. A voucher is the
equivalent of a ticket that can be used to make a reservation---the unit
of fulfillment.  When a
reservation is made against a particular voucher, it ``ties'' the
voucher to a specific show date.  If the reservation is cancelled, this
``tie'' is broken but the voucher itself remains, and can in some cases
be used to make  a new reservation later.

Each voucher is of a particular \T{voucher type}.  For each performance
(not just each production), the Box Office Manager specifies which
Voucher Types may be sold/redeemed for that performance, as well as any
applicable capacity controls or restricted sales dates on each type.  If
a particular voucher type is able to be redeemed for a particular
performance under some set of circumstances, we say that it is a
\T{valid voucher} for that performance under those circumstances.  In
other words, a Valid Voucher is really just a set of conditions that
specifies when a particular Voucher Type can be redeemed for a given
performance. 

A \T{bundle} is a collection of vouchers sold as a single unit.
A simple example is a subscription, which might contain three vouchers
valid for any musical and two vouchers valid for any play, or might
contain three different vouchers each of which is valid for a
performance of one specific production.  However, a bundle is not
automatically a subscription; bundles
can also be used for offering multi-ticket promotions that are not
subscriptions.  For example,  a ``family pack''
bundle could include 1 adult ticket and 1 child ticket, purchased
together for a discount price.  Adding a bundle product to a patron's
account causes the corresponding individual vouchers to be
added to the patron's account, after which point they behave just like
regular vouchers.  What's the difference between a bundle that is a
subscription and one that isn't?  The acquisition of a subscription
bundle qualifies the patron as a \emph{Subscriber}, and Audience1st
allows many operations to distinguish between Subscribers and
non-Subscribers.  For example, when a new Voucher Type is created, you
can specify whether it can be purchased by anyone or only by
Subscribers.  This makes it easy to offer premium tickets available only
for Subscribers, or a variation on general-admission tickets wherein
Subscribers can get a discount.  Lastly, recall that these distinctions
are only for purposes of ticket sales and fulfillment; in terms of
tallying your Accounts Receivable, each Voucher Type can be tagged with
an Account Code, and rollup reports can be generated based on these
keys.  So if you want to lump all of your ticket sales (both
Subscriber-only and general-availability) into a single income category
for accounting purposes, you'll just assign the same Account Code to all
those Voucher Types.

Confused yet?

A \T{ticket} is a special kind of voucher that is ``tied'' to a
particular performance \emph{at the time of purchase}.  Generally these
are cash purchases  by nonsubscribers to attend a  particular
performance.  

Vouchers, bundles and tickets are the \emph{products} offered for sale.
Each product can
be offered for sale to the general public, to
subscribers only, or to box office/administrators only.  For example,
courtesy tickets could be offered for customer service purposes, but
only the box office can add that type of ticket to an account.

\subsection{Patron's View}

All patrons have accounts by default; the first time someone makes a
purchase, Audience1st must collect an email address (to contact them
about their order) and a billing address for the credit card, and by
simply supplying a password, the patron now has an account (using their
email address as the login).  Online purchases \emph{require} the patron
to give an email address and supply a password.

For phone purchases, a Box Office Associate (BOA) may make a purchase on
the patron's behalf.  In this scenario, the BOA essentially walks
through the same steps the patron would walk through, creating the
account and then making the purchase.  The BOA can, at her option,
inform the patron that in the future the patron  may log in using the
provided email address and password.  (There is an option to leave the
login/email address blank when a BOA creates a patron account,
recognizing that some patrons phone in because they do not have email
access.) 

Once logged in, the patron sees a list of all their vouchers.  Some
vouchers may be marked as ``available'', meaning they have not yet been
``tied'' to a performance via a reservation.  Clicking on one of these
will allow the patron to make a reservation against that voucher.
A patron can also cancel a reservation, which makes the voucher
available for a different reservation.

Guests without accounts are directed to a \T{storefront} where specific
products are offered for credit card sale.  If the patron attempts to
purchase a subscription or bundle product, part of the purchase process
requires them to create an account.

\subsection{Backoffice View}
\label{sec:privilege-levels}

The various administrative users of the system are just patrons who have
special privileges.  They login and see the same screens that patrons
see, but they also have access to certain screens and operations that
patrons do not.  The levels in order of increasing privilege are:

\begin{enumerate}
\item[Patron] The default level: can log into her own account, manage
  her own reservations, and edit her own contact information.
\item[Staff] Can also generate reports, including mailing lists, box
  office statistics, etc.  Can record
  donations, search and update patron contact information, and manage
  the ``Resources'' database (patrons who are also professional service
  providers, professional connections to other organizations, etc.) This
  is the appropriate category for any 
  staff member who does not deal directly with reservation processing.
\item[Walkup Sales] can also do day-of-show box office procedures, such
  as generating the will-call list, processing walkup sales
  transactions, and generating the box office settlement report.
\item[Box Office Associate] Can also make and cancel advance
  reservations, search the patron database, update patron information,
  add and remove vouchers from patron accounts, and generate attendance lists for performances.
\item[Box Office Manager] Can add/edit shows, add/edit performance dates,
  add/edit voucher types and bundle types, determine which voucher types
  can be redeemed for which performances.
\item[Administrator] Can grant/revoke any of the above privileges to
  other users and perform additional direct manipulation of database tables.
\end{enumerate}

All backoffice actions are \T{audited}, with
permanent records being written of who performed a particular action and
when.  Thus, it is possible to distinguish (e.g.)  whether a patron
logged in and changed their own contact information or whether it was
changed by a Box Office Agent, and if so, who.

\subsubsection{Walkup Sales/Box Office Associate's View}

A Box Office worker who handles walkup sales has access to the necessary
functions for processing walkups.

A Box Office Associate (BOA) who also works with advance reservations
has additional capabilities:

\begin{itemize}
\item[Switch User] The BOA can search for a patron by name
  and then bring up that patron's record.  From that point on, the BOA
  can do anything the patron could do if the patron had logged in
  themselves.
\item[Edit User] The BOA can edit a patron's contact information or
  password, edit the private ``comments'' field (invisible to the
  patron) in the patron record, or record additional information about
  conversations or visits with the patron.  The BOA can \T{reset a
    patron's password}, but neither the BOA nor anyone else can see the
  current value of a patron's password.
\item[View Transactions] The BOA can view previous reservation/purchase
  transactions made by or on behalf of that patron.
\item[Record Donations] The BOA can record donations made by the
  patron. 
\end{itemize}

\subsubsection{Box Office Manager's View}

The Box Office Manager (BOM) can do everything the BOA can do, and can
additionally:

\begin{itemize}
\item view a patron's complete donation and purchase history
\item Add vouchers to a patron account without collecting payment
\item Remove/cancel prepaid vouchers from a patron account
\end{itemize}

The BOM can also add new shows and show dates.  Logically, a \T{show} is
a single production that has a run start date, a run end date, and some
other properties.  Each show is associated with one or more \T{show
  dates}, each of which represents a single performance.  In particular,
when a reservation is made against a voucher, the reservation ties the
voucher to a particular show date.

A show record includes information about the overall production,
including:
\begin{itemize}
\item Start and end dats
\item House capacity
\end{itemize}

A show date record includes various kinds of information such as:
\begin{itemize}
\item Date and time of the performance
\item Capacity of the house for the performance (if different from the
  default house capacity specified in the Show record)
\item When to start and stop advance sales for that performance
\item Which voucher types are valid for that performance, including the
  following information for each valid voucher type:
  \begin{itemize} 
    \item limits (if
      any) on how many vouchers of each types can be redeemed for that
      performance. For example, you may offer discount seats but allow at
      most 10 seats per show to be released as discount.
    \item Starting and ending date when reservations can be made for a
      particular show using this voucher type.
    \item Whether a promo code or password is needed to redeem this
      voucher type. 
    \item When to start and stop advance sales for this voucher type, if
      different from the default in the Show Date record.  For example,
      you may wish to allow time-limited access to a discount ticket
      type, or define a ticket type that is available early to certain
      groups of customers.
    \item Who may purchase this type of voucher: anyone may
      self-purchase, Subscribers 
      only may self-purchase, Box Office agent must handle the purchase,
      third-party vendor handles the purchase. The last choice is used
      for automatic integration of tickets offered through third-party
      outlets such as GoldStar Events or Tix Bay Area.
  \end{itemize}
\end{itemize}

\subsubsection{Administrators' View}

An Administrator (admin) can do anything, including assign any of the
above privilege levels to other users.  Admins can also do
operations that are
potentially dangerous to the database.  For example, permanently
deleting a patron is ``dangerous'' because past donations made by that
patron would no longer be linkable to an individual.  The administrator
is also the only privilege level that can add or remove privilege for
other users.

\subsection{Basic Navigation: Yellow Means Privilege}
\label{sec:navigation}

The basic navigational aid in \af is the use of color.

Elements in \emph{yellow} on the screen are ones
that you are seeing because of your administrative
privilege, but that a patron would not see.

For example, when a patron views her own account, she has the ability to
make or cancel reservations, but \emph{not} the ability to add or remove
vouchers from her own account (except by purchasing through the
Storefront).  Howver, when a Box Office Agent is viewing that same
patron's account, a yellow button labeled ``Add Vouchers''
will be displayed.  The yellow means that the patron would not see this
element but the box office associate does see it.

The most obvious yellow navigational aid is the \T{main navigation bar}
(navbar), 
which appears at the top of every screen for backoffice staff but is never
seen by patrons.  The navbar gives access to the  main functions of \af:
Customers, Reservations and Donations (chapter~\ref{chap:boa}), Shows and Show Dates
(chapter~\ref{chap:bom}), Transaction Search (chapter~\ref{chap:boa}),
Tickets and Subscription Types (chapter~\ref{chap:bom}), Reports
(chapter~\ref{chap:staff}), and a button to Log Out of the system.


\subsection{Logging In}
\label{sec:login}

All users---patrons or otherwise---see the same interface for logging
in.  To \T{login}, go to \url{http://www.audience1st.com/\venue} and enter your
login (email address) and password.  Note that since ``back office''
users are just patrons with extra privilege, backoffice staff wishing to
make reservations on behalf of a patron should just login as
themselves---not as the patron.  Once logged in, they will be able to
access accounts of other patrons.

\subsection{Test System}
\label{sec:test}

There is a \T{test system} available for backoffice users
that allows you to safely do operations on a ``sandbox'' database that
is completely separate from the real database.  You need to have a
separate login to use the test system (though you can choose to make it
the same as your login for the real system).

The test system is for
training and testing purposes and is always accessible at
\url{http://www.audience1st.com/\venue-test}. It is protected by a master
username and password (which you'll have to enter \emph{before} you see
the login screen).  Ask your system administrator what the master
username and password is.

You can tell when you are using the test system because there will be a
bright red border around every page. This border is absent when using
the production system.

TBD: how to distinguish build number of development system from build
number of production system.

\subsection{Direct SQL Access}
\label{sec:sql}

For administrators who are experienced in the direct use of SQL, access
is provided to the raw underlying SQL database for generating
arbitrarily complex reports.  If you don't know what SQL stands for,
this feature's not for you.

To enable raw SQL access, contact Audience1st and let them know the IP
address(es) from which you will establish a remote SQL client
connection.  You will be given a SQL username and password that has only
SELECT privilege on your databases.  The schema is described in more
detail in section~\ref{sec:raw-schema}.  Additional licensing
restrictions apply to the use of the raw SQL feature.

\subsection{Incorporating Into Your Venue's Web Site}
\label{sec:external}

AudienceFirst automatically provides an RSS~2.0-compliant feed of ticket
information for upcoming shows at
\url{http://www.audience1st.com/VENUE/store/ticket_rss}.  Your patrons
can subscribe to this feed directly, or you can display the feed info on
your site using (e.g.) the WordPress content management system or any
other CMS that has an RSS widget.  Each feed item reports on one
performance and reports whether it is sold out, nearly sold out (more
than 90 percent), or available, as well as whether tickets are available
in advance or only at the box office (advance sales have ended).

AudienceFirst also provides a VoiceXML call-in application that speaks
this same information to a caller and then hangs up.  The information is
limited to ticket info for shows between the time of the call and the
end of the coming weekend.  To connect this feature, users should call
((@@@phone number TBD@@@)), or be forwarded to that number via your
call-in switchboard.

