\section{Box Office Associate's Guide}
\label{chap:boa}

This chapter covers basic box office procedures including making and
cancelling reservations, recording donations, and day-of-show box office
procedures (generating attendance lists, etc.).  You must have at least
Box Office Associate privilege to access these screens and perform these
tasks. 

\subsection{The Patron Database Is Golden}
\label{sec:patrondb}

Our list of patron information is one of our biggest assets.  A few
words about its stewardship are therefore in order.

\textbf{Mailing addresses:} Patron address information is used for doing
mailings. When adding or updating a patron record, if you are not sure
about a patron's address, leave it blank. \emph{Don't} fill in ``N/A''
or ``Unknown'' or something like that; just leave it empty.  The same
goes for the email address.

\textbf{First and Last Names:} Patron names are used both for mailings
and for communication with patrons.  If a couple wants to use different
first and last names and want to keep their records distinct, then
create separate records.  If John Smith and Carol Jones are a couple and
they're OK having the same last name in our database, ``John and Carol''
for first name and ``Smith'' for last name is fine.  But if they want
their last names distinct, \emph{do not} enter ``John/Carol'' for first name
and ``Smith/Jones'' for last name---create two separate records.

\textbf{Shared addresses:} Similarly, when two or more patrons share a
household, make every effort to ensure that the addresses are really
\emph{identical} in the database.  In the future, we will have an
``address normalizing'' function that converts addresses to standard
US~Postal Service format, but for now, please be aware that to save
money on mailings we often perform de-duplication of addresses, which
will only work if the addresses are identical.

TBD: Validation of mailing address, de-duplication/merging procedure.

\textbf{If you make a mistake} while entering a transaction, see
section~\ref{sec:recovering} for common procedures to recover from mistakes.

\subsection{Finding,  Adding, or Changing Patron Account Info}
\label{sec:boa-admin}

\subsubsection{Looking Up a Patron or Determining Whether a Patron is a Subscriber}
\label{sec:lookup}

\begin{enumerate}
\item In the master navigation bar, click \emph{Customers \&
    Reservations}.
\item In the \emph{Search/Filter} box, type a single word that would
  narrow down to the desired patron.  (In the future, multi-word and
  advanced searches will be supported.)  This single word (or part of a
  word) can be something that occurs in the customer's first or last
  name, address, city, or any other contact info field.
\item If you see a match for the patron in the displayed list, you can
  click on the green arrow to go to that patron's record.  Note that if
  there are many matches, the ``Next Page/Previous Page'' links can be
  used to browse the search results.
\item If you're sure the patron is not in the database, see
  section~\ref{sec:addpatron} below to add them.
\end{enumerate}

The patron's general information screen shows their contact information,
whether they are a subscriber (``Valued Subscriber'' appears below their
name if so), and what tickets or reservations are currently held for
that patron.

\subsubsection{Adding a New Patron}
\label{sec:addpatron}

\begin{enumerate}
\item In the master navigation bar, click \emph{Customers \&
    Reservations.} 
\item Scroll to the bottom of the screen and click  \emph{New
    Customer}.
\item \emph{Carefully} fill in the new patron info.  You must choose a
  password for the patron; it's fine to use their last name, but for
  security purposes it can't be left blank.
\item Once patron info is entered, click the \emph{Create} button and
  you'll be taken to the (new) patron's account.
\end{enumerate}

\textbf{Note:} Email mailings are handled by a separate system called
PHPlist.  See section~\ref{sec:phplist} to understand how it is
connected to \af.

\subsubsection{Changing Patron Information or Password}
\label{sec:changeinfo}
\label{sec:changepassword}

You can  change a patron's contact information, including their login
(email address) and password.

Look up the patron (section~\ref{sec:lookup}) and at the bottom of the
patron's account page,  click \emph{Update Contact Info} or \emph{Change
  Password} as appropriate.  Enter the new information (if changing
password, you'll have to type the new password twice, for confirmation).

\textbf{Note:} For security reasons, patron passwords are stored in a
one-way hash format, which means that no one, including the
administrator of the whole system, can retrieve a patron's password.  We
can reset the password for them, but we cannot tell them what their
current password is.

\textbf{Note:} If the patron's email address has changed, or you want to
make changes to the patron's email preferences, you must take additional
steps; see section~\ref{sec:phplist}.


\subsubsection{``Blacklisting'' a Patron From Mailings}
\label{sec:blacklisting}

If a patron wishes to be removed from our US Mail list, simply edit the
patron's contact info (section~\ref{sec:changeinfo}) and check the
``Blacklist (don't include in mailings)'' box.

\subsubsection{Merging Duplicate Accounts}
\label{sec:merging}

If two patron accounts appear to be duplicates, they can be merged.
When they are merged, all vouchers, reservations, and donations of the
two merged accounts are assigned to the single remaining account.

To merge two accounts:
\begin{enumerate}
\item Display or search for the two accounts such that both are
  displayed on the screen at the same time.  For example, if the two
  accounts to be merged share a last name, you could search by that last
  name.
\item Check the box to the left of each of the \emph{two} patron
  accounts to be merged.  (To merge more than two accounts, you must
  merge them two at a time).
\item Click the \emph{Merge Selected Accounts} button.
\item On the next screen, for each field of patron information you will
  select which of the two accounts being merged should be used for that
  data.
\item When you've selected all the right data, click \emph{Merge}; to
  exit without doing anything, click \emph{Cancel}.
\item Note that the patron's login (email) and password are among the
  fields to choose for merging.  The only valid login and password for
  the patron after merging will be whichever login and password you
  selected on the merge fields screen.  If the patron complains of being
  unable to login, you can reset their password
  (section~\ref{sec:changepassword}). 
\end{enumerate}


\subsubsection{Managing Patron's Email Preferences}
\label{sec:phplist}

Email blasts are handled by a separate system called PHPlist.
Customer records are automatically linked with PHPlist records.

\textbf{When you enter a new customer}: if you supply an email address
for the new customer, \af will first check whether that email address
has already been registered in PHPlist.  (Possibly the patron added
herself to our email list \emph{before} becoming a full-blown patron.)
If yes, then the \af patron record will automatically be linked to the
corresponding PHPlist record.  If no, a new PHPlist record will be
created for the patron at the same time that the patron record is
entered in \af.

\textbf{When you update a customer's contact info:} if the patron has an
existing PHPlist record, you will see a button beneath the patron's
email address that says \emph{Update Email Prefs}.  Clicking this button
will open a \emph{new window} and take you directly to the PHPlist
record for this customer.  On this screen, you can change which email
lists the patron is subscribed to, whether the user should receive
HTML-formatted emails, and the user's email address.

If you're updating the customer's contact info and you are adding an
email address where previously no email address existed for the
customer, a new PHPlist record will be created.  

\textbf{Important.}  The customer's email address as used for email
blasts \emph{is distinct from} the customer's login on our system.
Suppose Joe Customer's current login on our system is \verb+joe@my.com+,
and you receive a request to update Joe's email information.  You visit
Joe's record, click the \emph{Update Email Prefs} button, and change his
email address in PHPlist to \verb+joe@your.org+.  From now on, any email
blasts that we send will correctly go to \verb+joe@your.org+.  But, when
Joe wants to login to our system, he will still have to login as
\verb+joe@my.com+.  If the customer wants the login changed as
well, explain that the login is only used to allow them access to the
boxoffice system---\emph{not} as the email address to send to.  If they
still really want the login changed, contact an administrator.


\subsection{Non-Revenue Reservations}

This section applies to reservations that do not generate income at the
time they are placed, including redemption of subscriber vouchers,
redemption of Comps, etc.

\subsubsection{Making a Reservation}
\label{sec:reservation-details}

Making a reservation means selecting an available voucher in a patron
account and selecting a performance to tie it to.  The Box Office
Manager sets up which vouchers can be redeemed for which show dates.
Possible criteria include:

\begin{itemize}
\item Advance sales start and cutoff: reservations can't be made before
  the start of advance sales or after the cutoff of advance sales.
\item Capacity reached for voucher type: Some types of vouchers, such as
  promos, may be capacity-controlled, e.g. only 20 discount seats can be
  sold for a given show.  Some promo vouchers may not be valid at all
  for certain shows, e.g. matinee-only vouchers can't be used to reserve
  for an evening performance.
\item Capacity reached for show: the voucher would be valid for the
  performance, but the performance is sold out.
\end{itemize}

Suppose patron John Doe has a ``valid for any musical'' voucher and a
``valid for any play'' voucher, and wishes to make a reservation for
\emph{Urinetown, The Musical}.  If John selects the ``play''
voucher, he will not be able to use it to reserve for \emph{Urinetown},
but if he selects his ``musical'' voucher, then \emph{Urinetown} will be
listed as a choice.

To make a reservation for a patron:

\begin{enumerate}
\item Locate the patron record or create a new one if needed
  (section~\ref{sec:boa-admin}). 
\item The patron record main screen shows a list of the vouchers in the
  patron's account.  The ``redeemed for'' column shows what performance
  each voucher is holding a reservation for.  If a given voucher is not
  currently being used to reserve for a given performance, it is shown
  as ``Available.''  \textbf{Exception:} A ``bundle'' voucher, such as a
  subscription, is a placeholder that is not reservable.  Instead, the
  individual vouchers that form part of the bundle appear in the list as
  reservable. 
\item Select an available voucher to use  by  clicking \emph{Make
    Reservation} to the right of the voucher.
\item Select a performance from the popup menu. Performances for which
  the voucher cannot be redeemed are shown in dimmed type and cannot be
  selected.  Each such performance also shows an explanation of why the
  voucher cannot be redeemed for that performance.
\item Select any special seating needs from the second popup menu.
\item Click \emph{Confirm Reservation} to make the reservation, or
  \emph{Don't Make Reservation} to cancel the action.
\end{enumerate}

\subsubsection{Cancelling a Reservation}
\label{sec:cancelling}

Note that a reservation cannot be cancelled if it is later than the
cancellation deadline (typically, 1 to 2 hours before curtain).


\begin{enumerate}
\item Locate the patron record or create a new one if needed
  (section~\ref{sec:boa-admin}). 
\item Identify the voucher in the patron's account whose reservation you
  want to cancel.
\item Click \emph{Cancel Reservation} to the right of the voucher.  A
  confirmation dialog will ask you to confirm if you really want to do
  this.
\item The voucher should then reappear as ``Available'' for future
  reservations. 
\end{enumerate}

\subsubsection{Changing a Reservation}

A reservation cannot be changed per se.  To ``change'' a reservation,
just cancel the original one (section~\ref{sec:cancelling}) and make a
new one (section~\ref{sec:reservation-details}).

\subsubsection{Adding Vouchers to Patron Account}

\begin{enumerate}
\item  Locate the patron record or create a new one if needed
  (section~\ref{sec:boa-admin}). 
\item Click \emph{Add Vouchers}.
\item Enter the number of vouchers and select the voucher type from the
  popup menu.  Note that adding a Bundle type voucher will add the
  correct number of individual Vouchers in the bundle.
\item Select the appropriate method to indicate how the voucher was
  acquired (customer purchase, courtesy/comp, etc.)
\item \textbf{Important.}  If you are entering an order that requires a
  separate fulfillment step, such as mailing an item to the customer,
  check the \emph{Fulfillment needed?} box.  This way the order will
  show up on the Unfulfilled Orders report (section~\ref{sec:report_unfulfilled}).
\item If necessary, add an optional comment relating to the order. For
  example, you can use this to enter a check number if the order was
  paid by check.
\end{enumerate}

\subsubsection{Removing Vouchers from a Patron Account}

Follow the procedure for making a reservation
(section~\ref{sec:reservation-details}, but instead of clicking
\emph{Make Reservation}, click \emph{Remove Voucher}. Note that if the
voucher is tied to a reservation, the reservation must be cancelled
before the voucher can be removed.

\subsubsection{Recording a Donation}

\begin{enumerate}
\item  Locate the patron record or create a new one if needed
  (section~\ref{sec:boa-admin}). 
\item Click \emph{Record Donation}.
\item Populate the form with the date of the donation, dollar value or
  amount of the donation, type of donation, destination for the
  donation, and any appropriate comments.
\item Note that you can add a new donation type or new fund (destination
  for the donation) by clicking \emph{Add New} next to ``Donation type''
  or ``Donation fund'' respectively.
\item Click \emph{Record} to record the donation or \emph{Back} to
  cancel. 

\end{enumerate}

\subsubsection{Transaction Search}
\label{sec:txnsearch}

If you are looking for a transaction associated with a particular
patron, the easiest way to find it is:

\begin{enumerate}
\item Look up the patron record (section~\ref{sec:lookup})
\item Click the yellow \emph{List Transactions} button on the patron
  info screen; it is next to the buttons for \emph{Add Vouchers} and
  \emph{Record Donations}
\end{enumerate}

\subsection{Revenue Sales/Reservations}
\label{sec:advance_sales}

This section describes how to process purchase of regular (revenue)
tickets by phone.

(TBD)

\subsection{Day of Show Procedures}
\label{sec:dayofshow}

\subsubsection{Integration With External Ticketing (GoldStar, etc.)}
\label{sec:externalticketing}

In addition to our own ticket sales, we also offer tickets through
certain external resellers such as GoldStar Events.  Information about
these tickets is normally received from the external vendor and
automatically integrated into our system so that it shows up in the Box
Office Report.

If a customer claims to have a GoldStar ticket but does not appear on
the box office list, see section~\ref{sec:checkgoldstarsales} for how to
manually double-check the GoldStar list.

\subsubsection{Generating the Box Office Report (Will-Call List)}

Since all our ticketing is electronic, the box office report and
will-call list are one and the same.

To generate the box office report for a performance:

\begin{enumerate}
\item Go to the Reports screen by clicking the Reports button in the
  navigation bar (see section~\ref{sec:navigation}.
\item On the Reports screen, look for \emph{Detailed Box Office
    Report}.  From the menu next to that choice, select the show date
  for which you want a report, and click the \emph{Go} button.
\end{enumerate}

\subsubsection{Walkup Sales---Cash, Check, Credit Card}
\label{sec:walkupsales}

To sell a ticket to a walkup, login to the system using any account that
has Walkup Sales privilege or higher, and perform the following:

\begin{enumerate}
\item Click the \emph{Walkup Sales} button in the yellow nav bar.
\item On the Walkup Sales screen, select the show and performance date
  from the dropdown menus.
\item Select the quantity of each ticket type. Note that subscribers who
  are bringing guests get a 10\% discount; this is shown as ticket types
  ``Guest of Subscriber -- Adult'' and ``Guest of Subscriber --
  Student/Sr''.  (If the subscriber does not have their subscriber
  credentials  and you prefer to verify that they are
  indeed a subscriber, check their customer record as described in
  section~\ref{sec:lookup}.)
\item If the customer wishes to also make a donation, fill in the amount
  in the Donation field.
\item When all done, click Record Sale.
\item Accept cash or check payment, or if a credit card, run the charge
  using the blue POS machine.  (This will change as soon as our credit
  card processor provides us with the functionality to integrate our own
  swipe.) 
\end{enumerate}

\subsubsection{Walkup Sales---Subscriber Without Reservation}
\label{sec:walkupsubscriber}

If a season subscriber shows up without a reservation and wants to see
the show, use the procedure of section~\ref{sec:reservation-details} to 
enter a reservation against one of their remaining Subscriber vouchers.

If they have no available vouchers valid for this show, they can always
purchase tickets at the ``Guest of Subscriber'' rate for themselves
(section~\ref{sec:walkup}). 


\subsubsection{Closing the Box Office}

After the box office closes:

\begin{enumerate}
\item Any additional walkup sales that were done by cash or check can be
  entered as a single large Walkup Sales transaction.  For example, a
  single transaction can be entered for 8 adults, 4 seniors and 4 comps
  that were dealt with as walkups.
\item TBD:  How to record the number of no-shows.  This capability will
  be available soon.
\end{enumerate}
