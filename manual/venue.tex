
\section*{Appendix: Additional Procedures}

\textbf{NOTE:} For security reasons, some passwords are not listed in
this document. There is a list of relevant passwords posted in the
boxoffice.

\subsection{Day-of-Show Problems}

\subsubsection{Manually Check GoldStar Sales}
\label{sec:checkgoldstarsales}

If the walkup list doesn't show any GoldStar customers, but you have
customers who claim they \emph{did} purchase from GoldStar, you can
manually check the GoldStar list in case there was some problem
importing the list into Audience1st.

\begin{enumerate}
\item Go to the GoldStar supplier
    page at http://www.goldstarevents.com/supplier
\item If a username/password is needed, use username \emph{altarena} and
  password \emph{shore}.
\item Click on \emph{Recent Will-Calls}.
\item Click on the appropriate date. Usually there are options for
  either PDF or Excel download; usually PDF works better, but you can
  try both.
\end{enumerate}

\subsubsection{Can't Login to Audience1st}

There is also a ``generic'' boxoffice account that is authorized for
walkup sales ONLY, and only from the computers in the boxoffice.  Its
username and password are posted in the boxoffice.

\subsection{Process Only a Credit Card Charge When No Card Present}
\label{sec:processcreditcard}

At the moment, we can use the blue POS terminal to swipe credit cards
for arbitrary amounts.  However, if you need to process a
card-not-present (e.g. telephone order) charge, you can use the Virtual
Terminal:

\begin{enumerate}
\item Go to our payment gateway website, \url{https://account.authorize.net}
\item Log in with the appropriate username and password posted in the
  boxoffice. 
\item Click the ``Tools'' button near the top of the screen. This will
  take you to the Virtual Terminal transaction screen.
\item Fill in the fields:
  \begin{itemize}
  \item Required: customer name, billing address
    (including country ``USA''),
    card number \& expiration, V-code.  
  \item You \emph{do not} need to fill in Shipping Address.
  \item Optionally, you can put a comment in the ``Description'' field
    (e.g. ``Gift certificate, 2 adult tix'').
  \end{itemize}
\item Click \emph{Submit transaction} to put the charge through.
\end{enumerate}



\subsection{Common Customer Questions}
\label{sec:commonquestions}

Here are some answers to questions that commonly come up.

If in doubt, or if the customer is not mollified, there is usually a
Board member in the theater before most performances.  If there is no
Board member present, pass the complaint on to one of the Board members
when you can, get the customer's contact information, and assure them we
will get back to them.  They also have the option of  calling the
theater's number and dialing extension number 4.  That is the ``escalate to
the Board'' unpublished extension for customer griping.

\subsubsection{How much are tickets? Is there a student discount? A
  senior citizen discount?}

General admission is \$20 adults, \$17 seniors (62 and over).  For
certain performances, we offer a limited number of Student Super
Discount tickets at just \$5 each.  When those tickets are unavailable,
the normal student price is \$17.  Students are up to age 21 (college
undergraduates) and will be asked to show a valid student ID at the
door.

\subsubsection{Is there a group discount?}

Yes, if it's a group of 10 or more. Have them call the theater and leave
a message, or email reservations@altarena.org, to learn about buying
blocks. 

\subsubsection{Can my group sit together?}

If you specifically do a block purchase (see above), we will hold seats
together for your block.  Otherwise, we are general seating on a
first-come first-served basis and \emph{we cannot guarantee you will be
  able to sit together}.  Note that we allow our Subscribers to be
seated before any non-Subscribers are seated.  Your best bet is to
arrive early; doors open one hour before curtain and seating begins 30
minutes before curtain.

\subsubsection{Are your  shows family-friendly? Are they lighthearted or serious?}

(Updated March 15, 2007)

For our 2007 season, the guidelines are:

\begin{itemize}
\item Virginia Woolf is rated PG-13 for adult situations.  It is a
  dramatic play and includes conflict.  It is considered an American
  masterpiece and the movie version of it won five Oscars.
\item The Last Five Years is suitable for preteens and older.  There is
  some mild language but the situations are rated PG.  It has a blend of
  happy moments and sad moments, just like real relationships.
\item Oh My Godmother is rated PG for some slightly risqu\'e humor.  It
  is an upbeat show with a happy ending.
\item Urinetown is rated PG to G.  It is an upbeat, fast-moving show
  with a lot of humor and satire, some of which may not make sense to
  children under 12 or so.
\item Morning's At Seven is rated G and is a gentle comedy with a happy
  ending. 
\end{itemize}

Unless otherwise noted on our Web site, all \emph{Spotlight Series}
concert events are family-friendly, designed to appeal to general
audiences, and appropriate for children ages 9
and up.  Events specifically suitable for even younger children will be
marked as such.

\subsubsection{I want to pay by check. You used to allow this.}

Checks are accepted in person at the Box Office only.  For advance
reservations, politely tell them they must either
\emph{mail} their check to us with a reservation request (and it may
take a week or more for us to get it, since we get a lot of mail),
or come to the Box Office in person 1 hour before showtime.
\emph{We cannot hold advance reservations that will be paid by check
at a later time.}  
(We did previously allow this in special cases but it was very rare.)

    
\subsubsection{I paid for tickets but I want to change the date}

Season Subscribers may change the date for free for any show included in
their subscription.  As a courtesy we ask that they do this 24 hours
prior to show time.

However, for \emph{paid} tickets and nonsubscribers, 
\textbf{NOTE:} It's OK for Season Subscribers to change their
reservations.  The AudienceFirst documentation covers that.  This
section refers to someone who has \emph{paid} for tickets and wants to
change.

Politely inform them that ticket sales are final.  (The BPT ticket sales
process makes this clear.)  

Suggestion 1: If they want to donate their unused tickets
to the theater, we would be happy to give them an IRS letter so they can
claim a tax deduction.  

Suggestion 2: They can give their tickets to a friend. 

We don't have the personnel to try to resell tickets for them.

\subsubsection{I want a refund}

We can't do chargebacks or refunds.  Sales are final.

If they're requesting a refund because they had a bad experience at the
theater or otherwise feel ripped off (as opposed to because they changed
their minds about wanting to go to the show), refer them to extension
4 and we will help them.


\subsection{Phone/Voicemail System Operations}
\label{sec:ringcentral}

RingCentral has fairly good online help, including online training
videos, user guides, and more. Just click the \emph{Get Help} button at
the top right of any RingCentral screen after logging in.

\subsubsection{Retrieving Voicemail}

\begin{enumerate}
\item Go to
  RingCentral at http://www.ringcentral.com/extensions
\item Login with phone number  5107649718, extension 2, passcode
  1234. You should reach a screen with a list of recent incoming calls.
\item To play a message, click the Play icon to its left.  You can
  replay a message as many times as you want.
\item To delete a message or mark it as Read (which preserves it as a
  Saved Message), check the box next to that message and click Delete or
  Mark As Read, as appropriate.
\item Note that some messages are marked with the Caller ID of the
  caller, if available; this is helpful if the message is garbled.
\end{enumerate}

\textbf{Note:} It is also possible to retrieve messages by phone, but
any phone time spent doing this is billed to us (per minute) by Ring Central.
Retrieving messages online is free.

\subsubsection{Changing Answering Rules}
\label{sec:answeringrules}

The answering rules determine how incoming calls are handled based on
the day, time, and (optionally) where the call is coming from.  Call
handling options can include forwarding to another phone number; ringing
multiple phone numbers at once (in case you don't know where the
answering party will be); or going straight to voicemail.  (If a call is
forwarded and there is still no answer, it will also go to voicemail.)

This section covers the basics of changing the answering rules; see
RingCentral's online help for more complicated scenarios.

Login as above to extension 2.
Near right side of screen under ``Quick Links'', click
  \emph{Answering Rules}.
The rules in the blue section of the screen are the ones you want
  to look at.  

Each rule has a ``When'' part, which says when that rule
should be triggered, and a ``what to do'' part.  Clicking on the
``When'' part lets you change the criteria of when this rule applies.
You basically specify a range of hours on each weekday.  (The rule
called ``Business Hours'' cannot be deleted or renamed, but it can be
changed.)

The ``what to do'' part has several options you can set for each rule:

\begin{itemize}
\item FindMe will cause the call to be forwarded to one or more
  additional phone numbers.  If FindMe is active, clicking on the phone
  number will take you to a screen where you can set up the FindMe
  criteria.  In particular, look at section 2 of that screen, ``Forward
  My Calls.''  You can list several phone numbers for forwarding, and
  specify whether they should be tried one at a time or all at the same
  time.  (The second option is for when you don't know whether the
  answering party will be at home or on their cell phone, e.g.)  If the
  call cannot be forwarded, it will go to voicemail.
\item If FindMe is shown as ``Not Configured'' for a particular rule, it
  means call forwarding is not done for that rule.
\item Take Messages Only will go straight to voicemail without
  forwarding the calls to anyone. (The voicemail will be left at this
  extension.)  \emph{Voicemail Greeting} specifies what greeting the
  caller will hear before leaving a message.
\end{itemize}

\subsection{Listing Tickets on Gold Star Events}
\label{sec:goldstar}

We usually offer a few tickets for 1/2 price on GoldStarEvents.com to
reach new audiences (GoldStar's emails reach over 30,000 people).  Our
usual policy is approximately:

\begin{itemize}
\item At least 10 tickets for opening night (more if sales slow)
\item At least 10 for first matinee
\item \textbf{NO} discount tickets for Actors Benefit (first Saturday)
\item About 6 tickets each for subsequent shows
\end{itemize}

These numbers can be adjusted if sales are particularly strong (or weak)
through our main channels.  The numbers can be adjusted after the
listing is sent to Gold Star, as long as we honor any tickets already
sold.

\textbf{There are two parts to listing GoldStar tickets for a show.}
First, you must ask GoldStar to list the tickets for us.  Second, you
must tell the AudienceFirst system to expect some GoldStar tickets.
Once these two steps are done, the will-call list sent from GoldStar
will be automatically received by AudienceFirst and integrated into our
box office listing.  If on the day of the show you have customers
claiming to have purchased through GoldStar but who aren't on the
will-call list, see section~\ref{sec:checkgoldstarsales} for how to
manually verify the GoldStar will-call list (in case there were problems
receiving it into AudienceFirst).

\textbf{NOTE:} Before you can submit a GoldStar listing, we must have a
page on our website (www.altarena.org) that describes the show.  You
will need the URL (Web address) of this page to complete this process.

\subsubsection{Listing Tickets With GoldStar}



\begin{enumerate}
\item Login to GoldStar as a
    supplier: \url{http://www.goldstarevents.com/supplier}.
\item Click Submit Listing.
\item Verify the contact information:
\begin{tabular}{r|l}
Company/Org: & Altarena Playhouse \\
Your Name: & Box Office manager's name \\
Email: & goldstar@altarena.org (use for all correspondence with
GoldStar) \\
Phone: & (510) 523 1553 \\
Event Title: & Name of show \\
Venue: & Altarena Playhouse \\
\end{tabular}

\item Dates and Times: enter a verbal description of the dates and times
  for which we want to offer discount tickets.  Usually we do this
  for the run of the show.

\item Offers: fill in only the first line of this table. For Seating,
  enter General Admission.  For Quantity, enter 6.  For Full Price,
  enter the individual ticket walk-up price for the show; for Goldstar
  Price, enter a number that is exactly half of the walk-up price.  (We
  do not distinguish senior/student discount on Goldstar.)

\item Event Web Page link: Enter the URL of the Altarena website page
  that decribes this show.

\item In the Notes box at the bottom, be sure to mention the following:
  \begin{itemize}
  \item NO discount tickets for first Saturday show (Actors Benefit)
  \item Please provide additional tickets (10 or 20, as determined by
    Box Office Manager, Artistic Director, etc.) for opening night and
    first Sunday matinee
  \end{itemize}

\end{enumerate}

\subsubsection{Entering GoldStar Allowance in AudienceFirst}

TO BE WRITTEN

\subsection{Email Aliases (you@altarena.org)}
\label{sec:emailaliases}

An email alias allows email to (someone)@altarena.org to forward to the
correct individual's personal email account.  Our organizational policy
is that there are two kinds of forwards:

\begin{itemize}
\item Individual staff members should have a forward that looks like
  \emph{firstname.lastname@altarena.org}, for example,
  \emph{armando.fox@altarena.org}. 
\item Positions/functions should have a forward that reflects the
  position or function, for example, \emph{auditions@altarena.org}.
\end{itemize}
                                             
To add, change or delete an email alias: 

\begin{enumerate}

\item  Login to our ISP administrative
    console at  \url{http://www.altarena.org/admin}  with the username and
  password supplied by the Webmaster. (This is deliberately different
  from the login info for the Altarena website's Internal section.)

\item In the left column, click on Email Aliases.�You'll see a list of
  existing aliases. 

\item To delete an existing alias, click its Delete box; to change who
  it forwards to, edit the email address in the right-hand column (to
  allow a single alias to forward to multiple individuals, enter all
  email addresses separated by spaces); or to add a new alias, fill in
  the new information. \emph{Don't forget to click Save Changes button
    at the bottom of the page when done. }

\item When done, click Logout button in left column.

\end{enumerate}

\subsection{Marketing Collateral}
\label{sec:marketing}

When creating graphics for a production, we typically need the
following:

\begin{itemize}

\item Postcard front---$4\times 6$ inches, hi-res (300 DPI or better),
  PDF, color (CMYK process).  Must include the following:
  \begin{itemize}
  \item Altarena logo in upper left or top center.  Logos can be found in
    the file \verb+logos.zip+ in the AltarenaMarketing group.
  \item Run dates of the show (opening night to closing night)
  \item Writer(s), director(s) and, if appropriate, music director(s)
  \end{itemize}

\item $11\times 17$ color poster.  Must carry all the same elements as
  the postcard front, \emph{plus}:
  \begin{itemize}
  \item ``Presented by Special Arrangement with Music Theatre
    International'' (or whichever licensing agency is used), in small
    type, bottom center

  \item  Altarena Playhouse, 1409 High St.,
    Alameda,
    \verb+www.altarena.org+, 510--523--1553
  \end{itemize}

\item 72dpi Web graphic, approx. 300 pixels wide.  This should be just a
  show graphic without the production info, which  will appear as text
  on the website. This graphic is also used in the promo email.

\end{itemize}
