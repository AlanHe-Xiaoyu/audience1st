\section{Staff Guide}
\label{chap:staff}

The main page of interest to non-boxoffice staff is the Reports page,
which can be used to generate reports about donors, orders, show sales,
etc.  This page is reached by clicking the yellow \emph{Reports}
navigation button in the navbar.

\subsection{Viewing Subscription Statistics}
\label{sec:subscriptionstats}

The top section of the Reports main page summarizes the number of
subscribers of each type and the total number of subcribers as of
today's date. 

The \emph{Download as Excel} button is currently not working, but it
will allow a summary of report data to be downloaded as an
Excel-compatible file.

\subsection{Viewing Attendance Statistics}

Attendance statistics come in two flavors.  The \emph{summary digest} view
shows the total attendance at each performance, optionally broken down
by ticket type.  
The \emph{boxoffice report} view is a detailed list of each patron's
name and ticket type for a particular performance, and it is displayed
in a decoration-free window designed to be printer-friendly.

For the summary view, select whether you want a summary of All Shows
(for this calendar year), Current Show Only (whichever show is currently
in production, based on its start and end dates), or Current and Future
shows (till the end of this calendar year).

The resulting summary view will show a line per performance with the
total attendance for that performance.  By clicking the \emph{Show
  Details} link at the right edge of each line, you can see a breakdown
of the ticket sales for that performance by ticket type.  The detail
view can be made to disappear by clicking \emph{Hide Details}.

\subsection{Generating Mailing Lists}
\label{sec:report_mailinglist}

To generate a mailing list of all patrons, mark the criteria you want to
use:
\begin{itemize}
\item Limit to subscribers only: only customers who are subscribers will
  be included. A customer is a subscriber if her/his account contains
  any vouchers that are marked as ``qualifies buyer as a subscriber''
  (see section~\ref{sec:vouchertypes-details}).
\item Only entries with validated US Mail addresses: limits to entries
  for which the customer address is believed valid.  Customer
  information entered by staff is always marked as valid at the time of
  entry.  Self-entered information (e.g. when customer purchases online)
  is marked as valid when the credit card transaction succeeds with the
  customer's address.  Use this option if you're generating mailing
  labels.
\item Remove duplicate addresses: consolidates to one entry per
  household address rather than per person.  For example, if we have a
  husband and wife living at the same address but with separate entries
  in our database, only one or the other of them will appear in the
  listing if this option is used.  (Currently, there's no way to specify
  which one it should be.)
\item Sort by: if the list needs to be batched by zipcode, choose the
  first option, which sorts by zipcode and then within each zipcode
  sorts by last name.  The second option sorts the whole list by last
  name, using zipcode to break ties.
\end{itemize}

When you've selected the options you want, click \emph{Download Report
in Excel Format}, and you should see a dialog box that will allow you to
either save the report to your computer or immediately open it in
Excel.  (The report is actually a CSV or comma-separated values file,
which is readable by other programs like Word as well; but Excel is
usually the best way to deal with list-like data.)

\subsection{Show Orders Needing Fulfillment}
\label{sec:report_unfulfilled}

Currently there are no options for this report. Clicking this button
will show orders for which fulfillment was indicated at the time the
order was placed or entered by staff, but for which fulfillment has not
occurred. 

Note that if the same customer has ordered multiple items, one line per
item will be shown on this screen, even if all the items were purchased
together. The reason is that a single order may combine items that have
different fulfillment requirements and for which fulfillment must
therefore be tracked separately.

To mark an item as fulfilled (e.g., the tickets were mailed to the
customer), check the box next to that item.  When you've checked all the
boxes you want to, click the \emph{Mark Checked Items as Fulfilled}
button.

Note that it's OK to do this multiple times.  That is, if you mark just
a couple of items, click the \emph{Mark Checked Items as Fulfilled}
button, and the re-run this report, you can still see the remaining
unfulfilled orders.
